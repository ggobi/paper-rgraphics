\documentclass[12pt]{article}

\setlength{\textwidth}{6.5in}
\setlength{\textheight}{9.0in}
\setlength{\oddsidemargin}{0in}
\setlength{\topmargin}{-0.5in}
\newenvironment{references}{
  \begin{center} \textsf{REFERENCES} \end{center}
  \begin{list}{}{\topsep=0pt\parsep=0pt\baselineskip=20pt
   \leftmargin=1.5em\itemindent=-\leftmargin}}
  {\end{list}}

\begin{document}

\noindent [Reviewer]\\ 
Hadley \textsc{Wickham} and Dianne \textsc{Cook} \\ 
Iowa State University \\ 
Department of Statistics, ISU \\ 
102 Snedecor Hall \\ 
Ames IA 50011-1210 \\ 
1-515-294-3440 (voice) x-4040 (fax) \\ 
dicook@iastate.edu \\ 


\noindent [Book \#6090]\\ 
{\Large\sf R Graphics}
\begin{quotation}\noindent
 Paul \textsc{Murrell}.  
Boca Raton, FL: Chapman \& Hall/CRC, 2006.  ISBN: 1-58488-486-X. xix+301 pp. \$69.95(H).  
\end{quotation}\vspace{12pt}

\setlength{\baselineskip}{20pt}

\section{Overview}

``R Graphics'' is an excellent technical introduction to drawing
graphics with R. It pulls together information currently scattered
throughout various R documents and help pages. The organization and
writing is clear and coherent, especially welcome when dealing with
the intricacies of R graphics.  It serves as a very useful reference
book.

\section{What's in this book?}

The book starts with an introduction to R graphics including a gallery
of many different graphics made using R, demonstrating the power of R
for generating static, high quality plots. The example plots range
from basic statistical plots, to cleverly annotated plots, to
sophisticated 3D diagrams and even whimsical art pieces! It is a very
inspiring beginning and one feels well motivated to do battle with R.

The main material in the book is divided into two parts.  The first
describes the traditional, or base, graphics, while the second part
introduces the newer grid graphics system, including lattice graphics.
Grid was designed and written almost entirely by the author of the
book, so Dr Murrell speaks with clear authority here.

Base graphics was the first graphics system developed for R, and if
you have used R in the past they are likely to be an old friend (or
enemy). Base graphics has a simple metaphor: ink on paper. Just like
drawing with your pen, you cannot undo your mistakes, except to start
afresh. More formally, there is no representation of the graphics
independent of their presence on the plot so you can only add, not
edit or delete existing output. This makes base graphics simple and
easy to understand, but fundamentally limited.  This limitation is
best seen when trying to customise graphics, where you either need to
start from scratch or grapple with many arcane settings. ``R
Graphics'' provides an excellent summary of these details.

In terms of functionality, but not yet popularity, base graphics has
been superceded by grid. The explanation of grid is the strength of
this book.  The section begins with a description of lattice graphics,
an illustration of the power of grid.  Lattice is an implementation of
trellis graphics (Becker, Cleveland and Shyu, 1996), which provide an
easy way to produce small multiples of different subsets of a data
set. The plots typically share the same scales and allow one to
investigate relationships between two variables conditional on one (or
more) other variables.  Lattice graphics present a higher level of
abstraction than base graphics, but configuring lattice can be
difficult due to the multitudinous (378 at last count) and repetitive
options.  This book provides a handy reference to some of these
options as well as a brief discussion of annotating lattice plots.
The description of creating new plot types is briefer still, largely a
reflection of the limitations with lattice.

The framework provided by grid graphics has a number of advantages
over the ink on paper design of base graphics:

\begin{itemize}

\item Grid objects have an independent representation as R objects,
not just pixels on a screen (described in Chapter 7.3).

\item A system of viewports allows for extremely flexible layout
(described in Chapter 5.5).

\item A range of coordinate systems makes it easy to draw what you
want where you want it (described in Chapter 5.3).

\end{itemize}

While it is easy to draw directly to the screen with grid, its real
power lies in creating objects that can be drawn at a later time. This
allows much greater flexibility as objects can be extensively modified
and even deleted.  Unfortunately, of the few R packages that use grid,
even fewer return grid objects, which makes building on top of them
almost as hard as building on top of base graphics.  We hope that this
will change as more people read this book. With base (and even
lattice) graphics it is hard to write reusable graphical functions,
and this is the promise of the grid system.

An R package, {\tt ggplot}, released after ``R Graphics'', further
demonstrates this potential. It builds on grid graphics using the
principles described in the Grammar of Graphics (Wilkinson, 2005),
providing R users with a higher level compositional language and good
defaults for generating both basic and sophisticated trellised plots.

\section{What's not in this book?} 

``R Graphics'' is a technical book. It does not attempt to explain the
how or why of statistical graphics.  It covers the technicality of
plot production, not purpose, giving the reader enough rope to be
extremely creative or to fail spectacularly. It does, however, point
readers to other books where they might learn about good graphics.

The book does not describe interactive or dynamic graphics, such as
linked brushing. Many readers may have experienced the power of a
tight coupling between statistical analysis and graphics through
XLispStat (Tierney, 1991) or DataDesk (Velleman, 1988). These systems
enable the user to make plots of model diagnostics that are
dynamically linked to plots of the data, updating themselves as a
model changes. This is very useful for exploring data, but it remains
difficult to realise within R. Base graphics and grid graphics are not
designed with interaction in mind so appropriately ``R Graphics''
focuses exclusively on static and presentation quality graphics, and
the author points readers to recent developments in interactive
graphics from R.

The book's website provides an electronic version of the code used in
the book, but little else. It would be useful to add material on
features of grid that have appeared subsequent to the release of the
book. For example, a recent release includes facilities for converting
postscript files to grid objects.

The description of grid graphics is the most important part of the
book, explaining a new, more powerful, graphics engine. However, it is
not given sufficient weight, and does not provide enough general
examples which could be easily adapted to a reader's data. Our fear is
that most readers will not get past the traditional graphics part of
the book.  While the book provides a good summary of this system, it
may slow the needful death of a primitive and out-dated system.

\section{Recommendation}

This is an excellent book. Everyone who uses R to draw graphics (all R
users, we hope!) should have it open on their desk or at least on
their shelves! We'd especially encourage readers to get beyond base
graphics and carefully study the grid graphics engine.

\begin{flushright}\def\baselinestretch{1}
Hadley \textsc{Wickham} and Dianne \textsc{Cook} \\ 
\emph{Iowa State University}
\end{flushright}

\begin{references} 
\item R. A. Becker, W. S. Cleveland, and M. J. Shyu. ``The Design and Control of Trellis Display,'' \emph{Journal of Computational and Statistical Graphics}, 5:123–155, 1996.

\item Tierney, L. ``LispStat: An Object-Oriented Environment for Statistical Computing and Dynamic Graphics'', Wiley, New York, 1991.

\item Velleman, P.F. and Velleman, A.Y. ``DataDesk Professional'', {\tt http://www.datadesk.com}. Odesta Corporation, Northbrook, IL, 1988

\item Wilkinson, L.  ``The Grammar of Graphics.''. Springer, New York, 2005.

\end{references}

\end{document}


