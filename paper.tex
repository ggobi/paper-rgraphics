\documentclass[12pt]{article}

\setlength{\textwidth}{6.5in}
\setlength{\textheight}{9.0in}
\setlength{\oddsidemargin}{0in}
\setlength{\topmargin}{-0.5in}
\newenvironment{references}{
  \begin{center} \textsf{REFERENCES} \end{center}
  \begin{list}{}{\topsep=0pt\parsep=0pt\baselineskip=20pt
   \leftmargin=1.5em\itemindent=-\leftmargin}}
  {\end{list}}

\begin{document}

\noindent [Reviewers]\\ 
Dianne \textsc{Cook} \\ 
Iowa State University \\ 
Department of Statistics, ISU \\ 
2415 Snedecor Hall \\ 
Ames IA 50011-1210 \\ 
1-515-294-8865 (voice) x-4040 (fax) \\ 
dicook@iastate.edu \\ 
\\
Hadley \textsc{Wickham} \\
Rice University \\
*** Need address\\

\noindent [Book \#***]\\ 
{\Large\sf R Graphics (second edition)}
\begin{quotation}\noindent
 Paul \textsc{Murrell}.  
Boca Raton, FL: Chapman \& Hall/CRC, 2011.  ISBN: 1-4398-3176-2. xxvii+518 pp. \$***(H).  
\end{quotation}\vspace{12pt}

\setlength{\baselineskip}{20pt}

\section{Overview}

It has been 6 years since the first edition was published. A lot has
changed in the world of statistical graphics during this time, which
is reflected in the substantial new material in the second
edition. Major changes include a chapter specifically devoted to the
\texttt{ggplot2} package and a new {\em large} section describing the
many available graphics packages and tools in R.  In our
review of the first book we described this book as an excellent
technical introduction to drawing graphics with R -- the changes make
this book the essential and comprehensive guide to graphics in R.

%Main points:

%\begin{itemize}
%\item No color
%\item Like the section on ggplot2 (new)
%\item Reference to other graphics packages (new)
%\item Refer to the old version
%\end{itemize}

\section{What's in this book?}

The book is divided into four parts -- base R graphics, grid graphics,
the graphics engine and graphics packages. The first section on
traditional base graphics is effectively the same as the first edition, and the
only major change to the second section on grid graphics is the
additional chapter on \texttt{ggplot2}.  The section on graphics
engines describes different graphics devices, primarily. The major
addition is the fourth section giving a supermarket of details on
available graphics tools in R.

The book starts with an introduction to R graphics including a gallery
of many different graphics made using R, demonstrating the power of R
for generating static, high quality plots. The example plots range
from basic statistical plots, to cleverly annotated plots, and
whimsical art pieces! It is a very inspiring beginning and one feels
well motivated to do battle with R.

Base graphics was the first graphics system developed for R, and if
you have used R in the past they are likely to be an old friend (or
enemy). Base graphics has a simple metaphor: ink on paper. Just like
drawing with your pen, you cannot undo your mistakes, except to start
afresh. More formally, there is no representation of the graphics
independent of their presence on the plot so you can only add, not
edit or delete existing output. This makes base graphics simple and
easy to understand, but fundamentally limited.  This limitation is
best seen when trying to customize graphics, where you either need to
start from scratch or grapple with many arcane settings. ``R
Graphics'' provides an excellent summary of these details.

With the emergence of the \texttt{lattice} and \texttt{ggplot2}
packages the grid engine has super-ceded base graphics, in popular use.
The framework provided by grid graphics has a number of advantages
over the ink on paper design of base graphics:

\begin{itemize} \itemsep 0in

\item Grid objects have an independent representation as R objects,
not just pixels on a screen.

\item A system of viewports allows for extremely flexible layout.

\item A range of coordinate systems makes it easy to draw what you
want where you want it.

\end{itemize}

To program in grid graphics requires seemingly low-level code,
pushing and pulling viewports, for example. The packages build in grid
translate the programming requirements into higher-level and more
data-centric code. But as higher-level languages, at some point users will hit a limit to creative
flexibility -- they cannot do absolutely everything they desire
in these systems. For example,
one of the frustrations in working with \texttt{ggplot2} is that it is not possible to draw multiple plots in 
one graphics device. The grid foundation enables plots created with each package to be enhanced
with grid commands, such as creating two \texttt{ggplot2} plots in a single device, which the advanced user will find well-documented in this book.   The grid graphics package was designed and
written almost entirely by the author of the book, so Dr Murrell
speaks with clear authority in the explanation of programming graphics
with grid.

The two chapters on the major graphics packages built on grid,
\texttt{ggplot2} and \texttt{lattice} are concise and well-written
introductions. To learn more about these packages the reader is
directed to the recently published books explaining them in detail:
Wickham (2009) and Sarkar (2008).

The new section on graphics packages generally available for use in R, is especially 
comprehensive. Here are some highlights of the material in this section:

\begin{itemize} \itemsep 0in
\item A chapter on the package \texttt{vcd} describes making
plots of categorical data. 
\item Map and network drawing packages are
described. 
\item Some color scheme choice packages are described, including
\texttt{ColorBrewer}. 
\item There is a chapter on interactive and dynamic
graphics packages \texttt{rggobi} and \texttt{iplots}. 
\item How to make
animations using the \texttt{animation} package is described. 
\item Some of the packages are utilities for reading in different
types of images and outputting different types of formats, such as
\texttt{SVGAnnotate}. 
\item One chapter is devoted to writing GUIs to enable
some interaction with graphics.
\end{itemize}

\noindent What is not entirely clear in the description of these plotting packages is on what base they are built. We tend to think  most are written in base graphics.

\section{What's not in this book?} 

COLOR! For an important graphics book, it is unfortunate that it does not
have color. In truth, a lot of the book describes code and technical
details of plotting in R, for which color is not needed, but there is some attention given to color choice and use in
different parts and these would have been greatly enhanced by color
figures. It is boggling to us why CRC didn't automatically insist on color
production here, and couldn't find a way to offer it at much the same
price. Similarly sized books with color are sold for a similar price
to this one.

``R Graphics'' still does not attempt to explain the how or why of
statistical graphics, which is reasonable.  It covers the technicality
of plot production, not purpose, giving the reader enough rope to be
extremely creative or to fail spectacularly. It does, however, point
readers to other books where they might learn about good graphics, and
plotting for data analysis.

The book's website provides an electronic version of
the code used in the book.

\section{Recommendation}

This is an excellent book. Everyone who uses R to draw graphics should have a copy! 

\begin{flushright}\def\baselinestretch{1}
Dianne \textsc{Cook} and Hadley \textsc{Wickham} \\ 
\emph{Iowa State University and Rice University}
\end{flushright}

\begin{references} 
\item Wickham

\item Sarkar
\end{references}

\end{document}